%*******************************************************
% Prefazione 
%*******************************************************

\myChapter*{Preface}
\markboth{\spacedlowsmallcaps{Preface}}{\spacedlowsmallcaps{Preface}} 
\mtcaddchapter[\numberline{}\tocEntry{Preface}]

%\addcontentsline{toc}{chapter}{\numberline{}\tocEntry{Preface}}

Hypergeometric functions  are useful in many different branches of applied
mathematics, theoretical physics, statistics, etc.  Part of the  explanation for
this probably  lies in the fact that (\S~\ref{chap:fuchs}) integrals of any
second-order linear homogeneus ordinary differential equation of Fuchsian class
having precisely three given regular singular points in the extended complex
plane can ultimately be expressed in closed form in terms of hypergeometric
functions  (\S~\ref{chap:hyper}). Among them are, for instance, Legendre
functions, Jacobi polynomials, Chebychev polynomials, etc.  Other differential
equations which often are encountered  in practise  are not of this form, one
relevant example being that of Bessel.  It may happen  however  that such
equations are confluent forms of the hypergeometric equation and in that case
they can be integrated in finite terms by means of confluent hypergeometric
functions (\S~\ref{chap:confluent}). (Bessel equation belongs to this class of
equations.) All of this handout is traditional material. 

\smallskip
 
\noindent\textsw{\myLocation, \myTime}


\begin{flushright}
    %\begin{tabular}{c}
        \Large{\calligra\myName} 
    %\end{tabular}
\end{flushright}

