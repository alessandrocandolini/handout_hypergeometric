
% ********************************************************************
% General commands
% ******************************************************************** 

% un ambiente ad hoc per le citazioni
\newenvironment{citazione}%
  {\begin{quotation}\small\ignorespaces}%
  {\end{quotation}}

% [...] ;-)
\newcommand{\omissis}{[\dots\negthinspace]}

% indica a LaTeX la cartella dove sono riposte le immagini
\graphicspath{{./},{./Asymptote/}, {./Images/}}

% eccezioni all'algoritmo di sillabazione
\hyphenation{Fortran ma-cro-istru-zio-ne nitro-idrossil-amminico}


% un ambiente ad hoc per gli approfondimenti
\newenvironment{approfondimento}%
  {\begin{quotation}\small\ignorespaces}%
  {\end{quotation}}

% i.e.
\newcommand{\ie}{i.\,e.}
\newcommand{\Ie}{I.\,e.}
\newcommand{\eg}{e.\,g.}
\newcommand{\Eg}{E.\,g.} 

% th
\newcommand{\ordth}{\textsuperscript{th}}


% ********************************************************************
% hyperref
% ******************************************************************** 
\hypersetup{%
    colorlinks=true, linktocpage=true, pdfstartpage=1, pdfstartview=FitV,%
    breaklinks=true, pdfpagemode=UseNone, pageanchor=true, pdfpagemode=UseOutlines,%
    plainpages=false, bookmarksnumbered, bookmarksopen=true, bookmarksopenlevel=1,%
    hypertexnames=true, pdfhighlight=/O,%
    urlcolor=webbrown, linkcolor=RoyalBlue, citecolor=RoyalBlue, pagecolor=RoyalBlue,%
% uncomment the following line if you want to have black links (e.g., for printing)
% urlcolor=Black, linkcolor=Black, citecolor=Black, pagecolor=Black,%
    pdftitle={\myTitle},%
    pdfauthor={\textcopyright\ \myName},%
    pdfsubject={},%
    pdfkeywords={},%
    pdfcreator={pdfLaTeX},%
    pdfproducer={LaTeX con hyperref e ClassicThesis}%
}

\hypersetup{citecolor=webgreen}
\hypersetup{hyperfootnotes=false,pdfpagelabels}

\newcommand{\mail}[1]{\href{mailto:#1}{\texttt{#1}}}


% *****************************************************************************
% Matematica 
% *****************************************************************************

% AMSmath packages 
\usepackage{amssymb}
\usepackage{amsmath}

% comandi per gli insiemi numerici (serve il pacchetto amssymb)
\newcommand{\numberset}{\mathbb} 
\newcommand{\N}{\numberset{N}} 
\newcommand{\Z}{\numberset{Z}} 
\newcommand{\Q}{\numberset{Q}} 
\newcommand{\R}{\numberset{R}} 
\newcommand{\C}{\numberset{C}} 

% Dirac notation  (serve il pacchetto braket)
\usepackage{braket} 
\newcommand{\modul}[1]{\mathinner{\vert#1\vert}} 
\newcommand{\Modul}[1]{\left\vert#1\right\vert} 
\newcommand{\norm}[1]{\mathinner{\Vert#1\Vert}}
\newcommand{\Norm}[1]{\left\Vert#1\right\Vert}
\newcommand{\conj}[1]{#1^{*}} 
\newcommand{\adj}[1]{#1^{\dagger}}

% un ambiente per i sistemi
\newenvironment{sistema}%
  {\left\lbrace\begin{array}{@{}l@{}}}%
  {\end{array}\right.}

% pacchetto cool 
\usepackage{cool}
\makeatletter
\Style{%
ArcTrig=arc,
IntegrateDifferentialDSymb={{\operator@font d}},
DSymb={{\operator@font d}},
DDisplayFunc=inset,DShorten=true}
\makeatother

% vectors (boldface style, primes are printed also in boldface)
%\usepackage[veceuler]{utvec}
\newcommand{\myvec}[1]{\mathbold{#1}}

\def\vec#1{
   \@ifnextchar'{\p@vec#1}{\np@vec#1}
}
\def\np@vec#1{\myvec{#1}}
\def\p@vec#1#2{%
   \@ifnextchar'{\p@vec{#1#2}}{\myvec{#1#2}}
}

\MakeRobust\vec


% Riemann P symbol via cool
\newcommand{\PRiemann}[4]{%
\listval{#1}{0}%
\ifthenelse{\NOT \value{COOL@listpointer}=3}%
{%
\PackageError{cool}{`PRiemann' Invalid Argument}%
{Must have comma separated list of length 3}%
}%
{}%
\listval{#2}{0}%
\ifthenelse{\NOT \value{COOL@listpointer}=3}%
{%
\PackageError{cool}{`PRiemann' Invalid Argument}%
{Must have comma separated list of length 3}%
}%
{}%
\listval{#3}{0}%
\ifthenelse{\NOT \value{COOL@listpointer}=3}%
{%
\PackageError{cool}{`PRiemann' Invalid Argument}%
{Must have comma separated list of length 3}%
}%
{}%
\mathchoice{%
P\inbr{\!%
\begin{array}{cccc}%
\listval{#1}{1} & \listval{#2}{1} & \listval{#3}{1} & \\%
\listval{#1}{2} & \listval{#2}{2} & \listval{#3}{2} & #4 \\
\listval{#1}{3} & \listval{#2}{3} & \listval{#3}{3} & 
\end{array}%
    \!}%
   }%
   {%
P\inbr{\!%
{\listval{#1}{1} \@@atop \listval{#1}{2} \@@atop \listval{#1}{3}}%
{\listval{#2}{1} \@@atop \listval{#2}{2} \@@atop \listval{#2}{3}}%
{\listval{#3}{1} \@@atop \listval{#3}{2} \@@atop \listval{#3}{3}}%
\, , \ #4
     \!}%
   }%
   {%
P\inbr{\!%
{\listval{#1}{1} \@@atop \listval{#1}{2} \@@atop \listval{#1}{3}}%
{\listval{#2}{1} \@@atop \listval{#2}{2} \@@atop \listval{#2}{3}}%
{\listval{#3}{1} \@@atop \listval{#3}{2} \@@atop \listval{#3}{3}}%
\, , \ #4
     \!}%
   }%
   {%
P\inbr{\!%
{\listval{#1}{1} \@@atop \listval{#1}{2}}%
{\listval{#2}{1} \@@atop \listval{#2}{2}}%
{\listval{#3}{1} \@@atop \listval{#3}{2}}%
\, , \ #4
     \!}%
   }%
}


% *****************************************************************************
% Teoremi e definizioni 
% *****************************************************************************

\usepackage{amsthm}                       % Serve il pacchetto amsthm.

\makeatletter
\newtheoremstyle{classicdef}%             % Stile tipografico dei teoremi
{11pt}%                                   % Spazio che precede l'enunciato
{11pt}%                                   % Spazio che segue l'enunciato
{}%                                       % Stile del font dell'enunciato
{}%                                       % Rientro (se vuoto, nessun rientro;
%                                         % \parindent = rientro dei capoversi)
{\scshape}%                               % Font dell'intestazione
{:}%                                      % Punteggiatura dopo l'intestazione
{.5em}%                                   % Spazio che segue l'intestazione:
%                                         % " " = normale spazio inter-parola;
%                                         % \newline = a capo
{}%                                       % Specifica intestazione enunciato
\makeatother

%\theoremstyle{definition}
\theoremstyle{classicdef}
\newtheorem{theorem}{Theorem}[chapter]
\newtheorem{lemma}{Lemma}[chapter]
\newtheorem{definition}{Definition}[chapter]
\newtheorem*{homework}{Homework}
\theoremstyle{remark}
\newtheorem*{remark}{Remark}
\renewcommand{\qedsymbol}{\rule{.5em}{.5em}}



% *****************************************************************************
% biblatex 
% *****************************************************************************

\renewcommand{\nameyeardelim}{, }

\defbibheading{bibliography}{%
\cleardoublepage
\manualmark
\phantomsection
%\addcontentsline{toc}{chapter}{\numberline{}\tocEntry{\bibname}}
\mtcaddchapter[\numberline{}\tocEntry{\bibname}]
\myChapter*{\bibname\markboth{\spacedlowsmallcaps{\bibname}}
{\spacedlowsmallcaps{\bibname}}}}     

% *****************************************************************************
% caption
% *****************************************************************************
\usepackage{caption}                      % Fancy captions and more.
\captionsetup{format=hang,font=small}
\captionsetup[table]{skip=\medskipamount} 


% *****************************************************************************
% makeidx, multicol
% *****************************************************************************
\let\orgtheindex\theindex
\let\orgendtheindex\endtheindex
\def\theindex{%
	\def\twocolumn{\begin{multicols}{2}}%
	\def\onecolumn{}%
	\clearpage
	\orgtheindex
}
\def\endtheindex{%
	\end{multicols}%
	\orgendtheindex
}

\makeindex

% *****************************************************************************
% breqn
% *****************************************************************************
\usepackage{mathtools}                    % Add support for cramped,
					  % mathlap,etc.
%\usepackage{siunitx}                      % Add support to print SI units
\usepackage[euler]{flexisym}              % Add support to Euler font
\usepackage{breqn}                        % Breqn

\makeatletter
   \def\eqnumsize{\normalfont \Tf@font}      % Add support to Minion Pro
\makeatother

\setkeys{breqn}{labelprefix={eq:}}


% *****************************************************************************
% Migliorare il kerning dell'apostrofo coi font MinionPro
% *****************************************************************************
\makeatletter 
\catcode`\'=12 
\def\qu@te{'} 
\catcode`'=\active 
\begingroup 
\obeylines\obeyspaces% 
\gdef\@resetactivechars{% 
\def^^M{\@activechar@info{EOL}\space}% 
\def {\@activechar@info{space}\space}% 
}% 
\endgroup 
\providecommand{\texorpdfstring}{\@firstoftwo} 
\protected\def'{\texorpdfstring{\active@quote}{\qu@te}} 
\def\active@quote{\relax 
  \ifmmode 
    \expandafter^\expandafter\bgroup\expandafter\prim@s 
  \else 
    \expandafter\futurelet\expandafter\@let@token\expandafter\qu@t@ 
  \fi} 
\def\qu@t@{% 
  \ifx'\@let@token 
    \qu@te\qu@te\expandafter\@gobble 
  \else 
    {}\qu@te{}\penalty\@M\hskip\expandafter\z@skip 
  \fi} 
\scantokens\expandafter{% 
  \expandafter\def\expandafter\pr@m@s\expandafter{\pr@m@s}} 
\makeatother


%%%%%%%%%%%
% margini %
% %%%%%%%%%%
% Suggestions from ClassicThesis
% Palatino 	10pt: 288--312pt | 609--657pt
% Palatino 	11pt: 312--336pt | 657--705pt
% Palatino 	12pt: 360--384pt | 768pt
\areaset[current]{336pt}{750pt}
\setlength{\marginparwidth}{7em}
\setlength{\marginparsep}{2em}%


%%%%%%%%%
% altro %
%%%%%%%%%

\hypersetup{citecolor=webgreen}
\hypersetup{pdfstartpage=1}


%\usepackage{mdframed}
\usepackage{nicefrac}
\usepackage{calligra}
\usepackage{sidenotes}


